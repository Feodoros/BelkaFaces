\documentclass{beamer}
\usepackage{beamerthemesplit}
\usepackage{wrapfig}
\usetheme{SPbGU}
\usepackage{pdfpages}
\usepackage{amsmath}
\usepackage{cmap} 
\usepackage[T2A]{fontenc} 
\usepackage[utf8]{inputenc}
\usepackage[english,russian]{babel}
\usepackage{indentfirst}
\usepackage{amsmath}
\usepackage{tikz}
\usepackage{multirow}
\usepackage[noend]{algpseudocode}
\usepackage{algorithm}
\usepackage{algorithmicx}
\usetikzlibrary{shapes,arrows}
\usepackage{fancyvrb}
\newtheorem{rutheorem}{Теорема}
\newtheorem{ruproof}{Доказательство}
\newtheorem{rudefinition}{Определение}
\newtheorem{rulemma}{Лемма}
\beamertemplatenavigationsymbolsempty

\title[]{Анализ решений задачи детекции лиц на изображениях в сфере киберкриминалистики}
\subtitle[]{}
% То, что в квадратных скобках, отображается в левом нижнем углу. 
\institute[СПбГУ]{
Санкт-Петербургский государственный университет \\
Кафедра системного программирования }

% То, что в квадратных скобках, отображается в левом нижнем углу.
\author[Жилкин Федор]{Федор Игоревич Жилкин, группа 17.Б10-мм \\
  % У научного руководителя должна быть указана научная степень
  \and  
    {\bfseries Научный руководитель:} к. т. н., доцент Ю.В. Литвинов \\ 
  % Для курсовой не обязателен. Должна быть указана должность или ученая степень
  \and
    {\bfseries Консультант:} рук. отд. раз. ПО, ООО “Белкасофт” Н.М.
Тимофеев}

\date{14 мая 2020г.}

\definecolor{orange}{RGB}{179,36,31}

\begin{document}
{
% Лого университета или организации, отображается в шапке титульного листа
\begin{frame}
  \begin{center}
  {\includegraphics[width=1.5cm]{pictures/SPbGU_Logo.png}}
  \end{center}
  \titlepage
\end{frame}
}

\begin{frame}[fragile]
  \transwipe[direction=90]
  \frametitle{Введение}
  \begin{itemize}
    \item Изображения, полученные с помощью камер видеонаблюдения или с помощью мобильных телефонов, активно используются службами безопасноти по всему миру для поиска нарушителей или установления личности 
    \item Необходимо автоматизировать процесс нахождения лиц на изображениях
    \item Практически нет работ, сравнивающих решения задачи детекции лиц в сфере киберкриминалистики
  \end{itemize}
\end{frame}
            


% Обязательный слайд: четкая формулировка цели данной работы и постановка задачи
% Описание выносимых на защиту результатов, процесса или особенностей их достижения и т.д.
\begin{frame}
  \transwipe[direction=90]
  \frametitle{Постановка задачи}
  \textbf{Целью} Целью работы является сравнение существующих решений
                задачи детекции лиц людей на изображениях на основе трех
                ключевых факторов:
                \begin{itemize}
                    \item качество решения обнаружения лиц
                    \item быстродействие при работе на CPU
                    \item работа под платформой .NET
                \end{itemize}

  \textbf{Задачи}:
  \begin{itemize}
    \item выбрать датасет, который будем использовать для тестирования решений
    \item рассмотреть существующие решения детекции лиц
    \item отдельно рассмотреть решения, работающие под платформой .NET
    \item изучить применимые для работы алгоритмы детектирования лиц
    \item создать тестирующую систему для сравнения решений
    \item провести полное сравнение всех решений
    \item выбрать лучшее решение для платформы .NET
  \end{itemize}
\end{frame}
            
\begin{frame}[fragile]
\transwipe[direction=90]
\frametitle{Эксперимент}
\textbf{Исследовательский вопрос:} какое решение покажет наилучшие результаты, если сравнивать по быстродействию и качеству обнаружения лиц? \\
\textbf{Критерии отбора решений:} Результаты соревнования на выбранном датасете; решения, работающие под .NET \\
\textbf{Датасет для тестирования:} Wider Faces \\
\textbf{Измеряемые метрики:} Recall, Precision, F-мера, время работы 

\end{frame}

\begin{frame}[fragile]
  \transwipe[direction=90]
  \frametitle{Формулировки теорем}
  \begin{rutheorem}[Пифагора: геометрическая формулировка]
    В прямоугольном треугольнике площадь квадрата, построенного на гипотенузе, равна сумме площадей квадратов, построенных на катетах.
  \end{rutheorem}

  \begin{rutheorem}[Пифагора: алгебраическая формулировка]
    В прямоугольном треугольнике квадрат длины гипотенузы равен сумме квадратов длин катетов.    

    То есть, если обозначить длину гипотенузы треугольника через $c$, а длины катетов 
через $a$ и $b$, получим верное равенство: $a^2 + b^2 = c^2$.
  \end{rutheorem}

  \begin{rutheorem}[Обратная теорема Пифагора]
    Для всякой тройки положительных чисел a, b и c, такой, что $a^2 + b^2 = c^2$, существует прямоугольный треугольник с катетами a и b и гипотенузой c.
  \end{rutheorem}  
\end{frame}

\begin{frame}[t]
  \transwipe[direction=90]
  \frametitle{Апробация}
  \begin{itemize}
    \item На каком множестве тестов проводилась апробация
    \item Какие результаты показала апробация
    \item Желательно привести графики, иллюстрирующие полученные результаты
    \begin{itemize}
      \item У иллюстраций должны быть подписи, у графиков -- легенда, подписи к осям, например:
    \end{itemize}
  \end{itemize}
  \includegraphics[width=10cm]{pictures/dist.png}
\end{frame}


\begin{frame}
  \transwipe[direction=90]
  \frametitle{Результаты}
  \begin{itemize}
    \item Практически то же, что и на слайде с постановкой задачи, но в совершенной форме — что делал лично автор
    \item Четкое отделение результатов своей работы (особенно для коллективных работ)
    \item Формулировать глаголами совершенного вида в прошедшем времени (``сделано'', ``получено'')
    \item Обсуждение (ограничения, валидность, альтернативы)
    \item Не нужно слайдов типа ``Все'', ``Вопросы?'', ``Cпасибо за внимание''
  \end{itemize}

  \begin{itemize}
    \item Если результаты были представлены на конференции и опубликованы, это желательно указать. 
  \end{itemize}
\end{frame}

\end{document}
